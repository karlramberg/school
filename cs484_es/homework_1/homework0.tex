\documentclass[11pt]{article}
\usepackage[paper=letterpaper, left=1in, right=1in, top=1in, bottom=1in]
           {geometry}
\usepackage[parfill]{parskip}
\usepackage{amsmath}
\usepackage[siunitx]{circuitikz}

\newcommand{\problem}[1]{\textbf{Problem #1 ---} }
\newcommand{\answer}{\textit{Answer: } }
\newcommand{\amp}{\ampere}

\begin{document}
\thispagestyle{empty}

\begin{center}
{\large Embedded Systems}\\
Homework Chapter 0
\end{center}

\begin{flushright}
Karl Ramberg
\end{flushright}

\problem{0.1} If the probes of a multimeter are placed on either side of a
  1.2\si{\kilo\ohm} resistor and measure a voltage of 9\si{\volt}, how
  much current flows through the resistor?

\answer Ohm's Law says that $V = IR$.  Rearranging, we have $I = \frac{V}{R}$.
Using the given values we have:
\begin{align*}
I &= \frac{V}{R}\\
  &= \frac{9\si{\volt}}{1200\si{\ohm}}\\
  &= 0.0075\si{\ampere}\\
  &= 7.5\si{\milli\ampere}
\end{align*}

\problem{0.2} If a resistor $R$ bridges the 5\si{\volt} and ground
rails of a power supply, what is the smallest E12 value it could have
if we want no more than 1\si{\milli\ampere} of current to flow through
the resistor?

\answer Ohm's law says that $V = IR$. Rearranging, we have $R = \frac{V}{I}$.
Using the given values we have:
\begin{align*}
R &= \frac{V}{I}\\
  &= \frac{5\si{\volt}}{1\si{\milli\ampere}}\\
  &= \frac{5\si{\volt}}{0.001\si{\ampere}}\\
  &= 5,000\si{\ohm}\\
  &= 5\si{\kilo\ohm}
\end{align*}
This means that the smallest resistance we must have is 5\si{\kilo\ohm}.
The smallest E12 series resistor that guarantees 5\si{\kilo\ohm} is a 5.6\si{\kilo\ohm} resistor. 
The 5.6\si{\kilo\ohm} resistor in an E12 series guarantees the resistance will between 5,040\si{\ohm} and 6,160\si{\ohm}, both ends being 10\% less or more than 5.6\si{\kilo\ohm}. 

\problem{0.3} If a resistor $R$ bridges the 5\si{\volt} and ground rails
of a power supply, what is the largest E12 value it could have if we
want no less than 20\si{\micro\ampere} of current to flow through the
resistor?

\answer Ohm's law says that $V = IR$. Rearranging, we have $R = \frac{V}{I}$.
Using the given values we have:
\begin{align*}
R &= \frac{V}{I}\\
  &= \frac{5\si{\volt}}{20\si{\micro\ampere}}\\
  &= \frac{5\si{\volt}}{0.00002\si{\ampere}}\\
  &= 250,000\si{\ohm}\\
  &= 250\si{\kilo\ohm}
\end{align*}
This means that the largest resistance we can have is $250\si{\kilo\ohm}$.
The largest E12 series resistor to guarantee no more than 250\si{\kilo\ohm} is a 220\si{\kilo\ohm} resistor.
This kind of resistor will guarantee the resistance doesn't go above 242\si{\kilo\ohm}, 10\% more than the base value.

\problem{0.4} How much power is consumed by the resistor in
Exercise~0.2?

\answer The Power law says that $P = IV$.
Using the given values we have:
\begin{align*}
P &= I\times V\\
  &= 1\si{\milli\ampere}\times 5\si{\volt}\\
  &= 0.001\si{\ampere}\times 5\si{\volt}\\
  &= 0.005\si{\watt}\\
  &= 5\si{\milli\watt}
\end{align*}

\problem{0.5} How much power is consumed by the resistor in
Exercise~0.3?

\answer The Power law says that $P = IV$.
Using the given values we have:
\begin{align*}
P &= I\times V\\
  &= 20\si{\micro\ampere}\times 5\si{\volt}\\
  &= 0.00002\si{\ampere}\times 5\si{\volt}\\
  &= 0.0001\si{\watt}\\
  &= 1\si{\milli\watt}
\end{align*}

\problem{0.6} A common USB power supply can output 5\si{\volt} with a
current of 2.1\si{\ampere}.  If we connect a 10\si{\ohm} resistor
across a USB power supply, how much current will it
draw?

\answer Ohm's law says that $V = IR$. Rearranging, we have $I = \frac{V}{R}$.
Using the given values we have:
\begin{align*}
I &= \frac{V}{R}\\\
  &= \frac{5\si{\volt}}{10\si{\ohm}}\\
  &= 0.5\si{\ampere}
\end{align*}

\problem{0.7} How much power is consumed by the resistor in
Exercise~0.6

\answer The Power law says that $P = IV$.
Using the given values we have:
\begin{align*}
P &= I\times V\\
  &= 0.5\si{\ampere}\times 5\si{\volt}\\
  &= 2.5\si{\watt}
\end{align*}

\problem{0.8} The rectifying diode in a power supply has a constant
voltage drop of 0.7\si{\volt}.  If 15\si{\ampere} of current flows
through the diode, how much power is consumed by the diode?

\answer To find the power drawn by the diode, we can use the power law.
The power law states that $P = IV$.
Using the given values we have:
\begin{align*}
P &= I\times V\\
  &= 15\si{\ampere}\times 0.7\si{\volt}\\
  &= 10.5\si{\watt}
\end{align*}

\problem{0.9} The bands on a resistor are green-blue-yellow-gold.
What is the minimum and maximum resistance the resistor may have?

\answer Green, blue, yellow would correspond to the numbers $5$, $6$, and $4$ respectively.
The gold band indicates a 5\% tolerance. 
This means the resistor's base value is 560\si{\kilo\ohm}, 
the minimum resistance is $560\si{\kilo\ohm}\times 0.95 = 532\si{\kilo\ohm}$ 
and the maximum resistance is $560\si{\kilo\ohm}\times 1.05 = 588\si{\kilo\ohm}$.

\problem{0.10} The bands on a resistor are gold-brown-red-gray.  What is
the minimum and maximum resistance the resistor may have?

\answer This ordering is flipped because the gold band should be read last on the right;
gray-red-brown-gold would be easier to understand.
Gray, red, brown corresponds to the numbers $8$, $2$, and $1$ respectively.
Gold means the resistor has a 5\% tolerance.
The nominal value would be 820\si{\ohm},
the minimum resistance is $820\si{\ohm}\times 0.95 = 779\si{\ohm}$,
and the maximum resistance is $820\si{\ohm}\times 1.05 = 861\si{\ohm}$.

\problem{0.11} A surface mount resistor is marked 752.  What is the
nominal value of the resistor?

\answer The nominal value of this surface mount resistor would be 7.5\si{\kilo\ohm}.

\problem{0.12} A surface mount resistor is marked 051.  What is the
nominal value of the resistor?

\answer The nominal value of this surface mount resistor would be 50\si{\ohm}.

\problem{0.13} A surface mount resistor is marked 2201.  What is the
nominal value of the resistor?

\answer The nominal value of this surface mount resistor would be 2.2\si{\kilo\ohm}.

\end{document}
